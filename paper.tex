% This is a sample paper template for CS331 Assignment
% Using LLNCS macro package for Springer Computer Science proceedings
% Version 2.20 of 2017/10/04

\documentclass[runningheads]{llncs}

\usepackage{graphicx}
\usepackage{cite}
\usepackage[T1]{fontenc}
\usepackage{url}
\usepackage{booktabs}  
\usepackage{amsfonts}  
\usepackage{nicefrac} 
\usepackage{microtype}
\usepackage{lipsum}
\usepackage{amsmath}
\usepackage{xcolor}
\usepackage{enumitem}
\usepackage{tabularx}
\usepackage{subcaption}
\usepackage{caption}
\usepackage{floatrow}
\usepackage{gensymb}
\usepackage{tabularray}
\usepackage{float}
\usepackage{soul}
\usepackage[ruled]{algorithm2e}
\usepackage{subcaption}
\floatsetup[table]{capposition=above}
\captionsetup[table]{labelfont=bf}
\setlength{\intextsep}{5pt plus 2pt minus 2pt}
\newcolumntype{C}[1]{>{\centering\arraybackslash}m{#1}}

% If you use the hyperref package, please uncomment the following line
% to display URLs in blue roman font according to Springer's eBook style:
% \renewcommand\UrlFont{\color{blue}\rmfamily}

\begin{document}

\title{Efficient Generation and Analysis of Terminal Tic-Tac-Toe Positions on a 5×5 Board}

\author{Nguyen Dinh Khiem\inst{1}}
\authorrunning{Khiem}
\titlerunning{5×5 Tic-Tac-Toe Dataset Generation}

\institute{
\email{Khiem.ND227983@sis.hust.edu.vn} \\
Troy University \\
Lecturer: Assoc. Prof. Van Hai Pham \\ \email{haipv@soict.hust.edu.vn} \\
}

\maketitle

\begin{abstract}
This paper presents a novel approach to generating and analyzing terminal positions in 5×5 Tic-Tac-Toe games. We introduce an efficient method for constructing a comprehensive dataset of legal terminal positions, leveraging board symmetries for deduplication and employing a compact binary encoding format. Our approach uses reverse generation from winning lines, coupled with rigorous legality validation and canonical form representation, to create a complete corpus of terminal game states. The methodology ensures completeness while maintaining practical computational requirements through symmetry-aware optimizations and streaming data management. The resulting dataset provides valuable insights into game characteristics and serves as a foundation for further analysis of advanced Tic-Tac-Toe variants.
\end{abstract}

\keywords{Tic-Tac-Toe \and Game Analysis \and Combinatorial Games \and Symmetry Optimization \and Dataset Generation}

\section{Introduction}
\label{sec:intro}

Tic-Tac-Toe, despite its simple rules, presents interesting computational challenges when scaled to larger board sizes. This paper focuses on the 5×5 variant with a win condition of five-in-a-row, exploring methods to efficiently generate and analyze all possible terminal game states.

Our research addresses several key challenges:

\begin{itemize}
    \item Generation of all legal terminal positions from an empty board
    \item Efficient handling of board symmetries to eliminate duplicates
    \item Validation of game state legality and reachability
    \item Compact representation and storage of the dataset
\end{itemize}

The significance of this work lies in both its methodological contributions to game state analysis and its practical applications in game theory and artificial intelligence research.

\section{Related Work}
\label{sec:related}

While extensive research exists on standard 3×3 Tic-Tac-Toe and its variants, systematic analysis of larger board sizes has been limited by computational complexity. Previous work has primarily focused on forward game tree exploration, which becomes intractable for 5×5 boards due to the exponential growth in possible states.

Our approach builds upon:
\begin{itemize}
    \item Symmetry reduction techniques in combinatorial game theory
    \item Efficient board state encoding methods
    \item Reverse game state construction strategies
\end{itemize}

\section{Methodology}
\label{sec:methodology}

\subsection{Reverse Generation Approach}
\label{subsec:reverse}

Instead of exploring the forward game tree, we construct terminal states directly by working backwards from winning configurations. The process involves:

\begin{enumerate}
    \item Identifying geometric winning lines (5 contiguous cells)
    \item Selecting winner and feasible total ply count
    \item Placing winning marks and distributing remaining pieces
    \item Validating board state legality
\end{enumerate}

\subsection{Symmetry Optimization}
\label{subsec:symmetry}

We leverage the D4 symmetry group of the square board, comprising:
\begin{itemize}
    \item 4 rotations (0°, 90°, 180°, 270°)
    \item 4 reflections (horizontal, vertical, main diagonal, anti-diagonal)
\end{itemize}

This reduces the search space by identifying and storing only canonical representatives of equivalent positions.

\subsection{Data Encoding and Storage}
\label{subsec:encoding}

Our binary encoding scheme achieves efficient storage:
\begin{itemize}
    \item 2 bits per cell (00=empty, 01=X, 10=O)
    \item 25 cells → 50 bits total
    \item Packed into 64-bit words for alignment
\end{itemize}

\section{Results}
\label{sec:results}

\subsection{Dataset Characteristics}
The generated dataset exhibits several notable patterns:

\begin{itemize}
    \item Minimum ply counts: 9 for X wins, 10 for O wins
    \item Mid-game positions show highest diversity
    \item Late-game positions constrained by board filling
\end{itemize}

\subsection{Performance Analysis}
The approach demonstrates efficient resource utilization:

\begin{itemize}
    \item Streaming processing reduces memory requirements
    \item Symmetry optimization significantly reduces storage needs
    \item Parallelizable across different winning lines and plies
\end{itemize}

\section{Discussion}
\label{sec:discussion}

\subsection{Advantages}
\begin{itemize}
    \item Complete coverage of terminal positions
    \item Efficient storage through symmetry reduction
    \item Scalable processing through streaming
\end{itemize}

\subsection{Limitations}
\begin{itemize}
    \item Winner/ply information implicit in layer structure
    \item Single-threaded generation bottlenecks
    \item Limited to geometric symmetry deduplication
\end{itemize}

\section{Conclusion}
\label{sec:conclusion}

We have presented an efficient methodology for generating and analyzing terminal positions in 5×5 Tic-Tac-Toe. Our approach combines reverse generation, symmetry optimization, and compact encoding to create a comprehensive dataset of terminal game states. The resulting corpus provides a valuable resource for game analysis and AI research.

Future work could include:
\begin{itemize}
    \item Extension to larger board sizes
    \item Parallel processing optimizations
    \item Integration of semantic equivalence analysis
    \item Enhanced metadata storage
\end{itemize}

% Bibliography
\bibliographystyle{splncs04}
\bibliography{ref}

\end{document}